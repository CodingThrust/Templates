\section{Style and Formatting}
White A4 paper of good quality (80 to 90 gsm woodfree paper, i.e. photocopying
paper) is to be used. Lines should be one-and-a-half line spaced throughout the
thesis, except for abstracts, indented quotations or footnotes where single line spacing
may be used. Each page should begin with a new paragraph. The IEEE style\footnote{References should be in a separate section at the end of the thesis, with items referred to by
numerals in square brackets, using the IEEE style as follows:

Papers : Author(s) (first initials followed by last name), title, periodical, volume, inclusive
page numbers, month, year.

Books : Author(s) (first initials followed by last name), title, location, publisher, year,
chapter, page numbers.} is
recommended for the references. Reference citations within the text should be in the
form of numbers within square brackets (e.g. \citep{guidelines}).

All margins should be consistently 25mm (or a maximum of 30mm) in width. The
same margins should be used throughout a thesis. All preliminary pages (title,
approval, abstract, acknowledgement, etc.) should be numbered using lowercase
Roman numerals (e.g. i, ii, iii, iv, …) but i is not shown on the title page. On all other
pages use Arabic numerals, i.e. 1, 2, 3, 4, …

\section{Examples}
Here are some example pages with figures and tables. The content of the sample pages are extracted from the HKUST brand guidelines. Please remove all the sample content from your thesis.

\subsection{Figure}
Sample page: A page of illustrations with legends.
\begin{figure}[htb]
\centering
\includegraphics[width=0.5\columnwidth]{figure/fig_HKUST.png}
\caption{Standard University Combination Logo - English}
\label{fig_HKUST}
\end{figure}

Figure \ref{fig_HKUST} is the English version of the HKUST standard university combination logo. A combination logo consists of logotype (e.g. texts) and logomark (e.g. image).
This standard University combination logo should be used whenever possible
on all English communications only, including University printed materials,
presentations, websites, and University souvenirs. The standard University combination logo should be used in its entirety with no alterations or additional elements.

To make sure the standard University combination logo is always clear and legible, there is a minimum size requirement. 
The minimum size requirement is based on the height of the standard University combination logo.

\subsection{Table}
Sample page: A text page with a table 

\begin{table}[htb]
\caption{Symbolic meaning of the HKUST emblem.}
\centering

\begin{tabular}{|c|l|}
\hline Parts & Meaning \\
\hline
\multirow{4}{*}{As a whole}& A scholar holding up an open book \\ 
& A transmission tower \\
& - representing engineering and technology \\
& - or communication and management \\ \hline
\multirow{2}{*}{Top circle}& Golden head of wisdom of the scholar\\
& A sun radiating gold (very traditional Chinese color)\\ \hline
\multirow{5}{*}{Wave beneath the circle} & The open book of knowledge\\
& An ocean representing Clear Water Bay, Hong Kong \\
& - where the University locates\\ 
& Upsilon \\ 
& Top part of Tau \\\hline
\multirow{2}{*}{Curves on both side}  & Arms of a scholar\\
& Sides of Psi \\ \hline
\multirow{4}{*}{Bottom center}  & Body of the scholar\\
& Flask representing science\\
& Center part of Psi \\
& Bottom part of Tau \\ \hline
\end{tabular}
\label{tbl_wmt17-segrr}
\end{table}

The emblem of HKUST is in several ways symbolic of the institution. Lyrically it visualizes the golden head of wisdom over the open book of knowledge. Between the arms holding the book can be seen as a flask representing science. Alternatively, it is a transmission tower representing engineering and technology or communication and management. We can also see a sun radiating gold, that very traditional Chinese color, over an ocen glowing with the deep blue representative of Hong Kong. Supporting these emblems are the three Greek letters upsilon, psi, and tau, that is: transliterating as UST. The emblem entwines many meanings, as does the University itself.

